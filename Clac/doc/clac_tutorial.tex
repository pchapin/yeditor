% ===========================================================================
% FILE         : clac_tutorial.tex
% SUBJECT      : This is the tutorial for the CLAC program.
% LAST REVISED : 2007-11-30
% AUTHOR       : Peter C. Chapin
%
% This is the tutorial for the CLAC program. It attempts to teach the user about most of CLAC's
% more interesting features.
% ===========================================================================

% ++++++++++++++++++++++++++++++++
% Preamble and global declarations
% ++++++++++++++++++++++++++++++++
\documentclass{report}
\pagestyle{headings}
\setlength{\parindent}{0em}
\setlength{\parskip}{1.75ex plus0.5ex minus0.5ex}

% ===========================================================================
% FILE   : macros.tex
% SUBJECT: This file contains useful macros used in all the Clac documentation.
% AUTHOR : Peter C. Chapin
%
% ===========================================================================

\newcommand{\CLAC}{Clac}



% +++++++++++++++++++
% The document itself
% +++++++++++++++++++
\begin{document}

% ----------------------
% Title page information
% ----------------------
\title{\CLAC\ Tutorial}
\author{Peter Chapin\thanks{PChapin@vtc.vsc.edu}\\
        Peter Nikolaidis\thanks{peter@paradigmcc.com}}
\date{June 22, 2002}
\maketitle

% ------------------------
% Make a table of contents
% ------------------------
\pagenumbering{roman}
\tableofcontents
\newpage
\pagenumbering{arabic}

% -------------
% Chapter Break
% -------------
\chapter{Introduction}

\CLAC\ is a general purpose scientific calculator program. \CLAC\ also has features that would
be of interest to programmers and computer scientists. \CLAC\ offers the following significant
features:

\begin{enumerate}
  
\item \CLAC\ has a stack oriented design that makes it easy to evaluate complex expressions.
  
\item \CLAC\ supports all the basic scientific functions.
  
\item \CLAC\ allows you to use several different data types in your calculations. For example,
  \CLAC\ allows you to manipulate complex numbers, matricies, fractions, and huge integers as
  easily as floating point numbers.
  
\item \CLAC\ offers a tree structed directory for storing the results of calculations.
  
\item \CLAC\ is fully dynamic. It has few arbitrary limits. \CLAC\ is only limited by the size
  of available memory.
  
\item \CLAC\ is fully programmable. Using a language very similar to Forth, \CLAC\ supports
  conditional branching, loops, subprograms, arrays, pointers, structures, and more.

\end{enumerate}

We intend for \CLAC\ to be used primarily interactively. Although highly programmable, \CLAC\
programs tend to be slow and hard to read. Furthermore some operations are awkward to program.
We expect that you will use \CLAC\ programs to streamline your interaction with \CLAC. If you
want to do serious numerical work, you might use \CLAC\ as a prototyping tool prior to writing a
program in another language.

\CLAC\ does not support certain features commonly found on today's high end calculators. In
particular, \CLAC\ does not do symbolic manipulation or graphing. We may implement these
features in future versions of \CLAC.

In addition, \CLAC\ does not directly support a number of functions that are built into many
calculators. For example, \CLAC\ does offer any statistical or financial functions. You can
implement many of these functions yourself using \CLAC's programming language. This tutorial
demonstrates several possibilities.

This document is a complete tutorial on using \CLAC. First, we discuss the basics of calculating
on a stack machine. Next we discuss the data types and built-in operations that \CLAC\ supports
and what you can do with them. Finally, we discuss how to program \CLAC.

Remember that this document is only a tutorial. We do not describe \CLAC\ completely in this
document. See the reference manual for more detailed information.

% -------------
% Chapter Break
% -------------
\chapter{\CLAC\ as a Stack Machine}

To get the most out of this tutorial, you should print out this document and read it as you play
with \CLAC. There is nothing quite like trying things out as you go along.

\CLAC\ does all of its calculations (or ``claculations'' as we like to say) on a stack. Think of
a stack of books. The top book is on stack level one. The book right below it is on stack level
two. If you put a new book on the stack, the new book becomes stack level one. The old stack
level one becomes stack level two.

In \CLAC\ the items on the stack are numbers, not books. When you do a calculation, you must
first put numbers on the stack, and then operate on them.

\CLAC\ displays its stack in the stack window at the upper left of the screen. Stack level one
is always displayed at the bottom. As you put more numbers on the stack, the old numbers are
shifted up. This arrangement allows stack level one to always be at the same spot on the
display. In contrast with a stack of books the top book gets farther and farther from the table
as the stack grows. As you can probably guess, most of the action is at stack level one.

For example, when you start \CLAC, the stack looks like

\begin{tabular}{|r|} \hline
 0 \\ \hline
 0 \\ \hline
 0 \\ \hline
 0 \\ \hline
\end{tabular}

After you put the numer 1234 onto the stack, it looks like

\begin{tabular}{|r|} \hline
    0 \\ \hline
    0 \\ \hline
    0 \\ \hline
 1234 \\ \hline
\end{tabular}

If you then put 3.14 onto the stack, it will become

\begin{tabular}{|r|} \hline
    0 \\ \hline
    0 \\ \hline
 1234 \\ \hline
 3.14 \\ \hline
\end{tabular}


\CLAC\ only shows the first eight stack levels. However, \CLAC\'s stack is as large as it needs
to be until \CLAC\ runs out of memory. For example, if a number is shifted to stack level nine,
it will move back to stack level eight if the stack shrinks.

Suppose you wanted to add 2.17 and 3.14. To do this, you must first put the two numbers on the
stack. Since this is addition, the order doesn't really matter. At the \CLAC\ command prompt (in
the bottom window), type ``2.17'' (no quotes) and strike the return key. We say that you've
pushed 2.17 onto the stack. Next push 3.14 onto the stack by typing ``3.14'' and striking the
return key.

Now that the numbers of interest are on the stack, type ``+'' (again, no quotes) and strike the
return key. The + operation adds whatever numbers it finds on the top of the stack and replaces
stack levels one and two with the sum: 5.31. Notice that two items have been ``popped'' from the
stack and that one item has been pushed back on.

As it happens, \CLAC\ allows you to type the entire calculation on one command line. For example

\begin{verbatim}
     => 7.23 .41 +
\end{verbatim}

This command pushes 7.23 onto the stack, pushes 0.41 onto the stack, and then invokes the ``+''
operation. After completing the command, \CLAC\ should show you 7.64 in stack level one. As you
gain experience with \CLAC, you will probably find yourself typing longer and longer commands at
the prompt. However, if you are ever uncertain about what's going to happen, you can always
break a calculation down into small steps to make sure it's working as you expect.

Notice that the 5.31 calculated before is still there (now in stack level two). After all, the
5.31 may just be the first part of some larger calculation. Type ``+'' at the command prompt to
compute the sum of 5.31 and 7.64.

You can type ``clear'' at the command prompt to erase everything in the stack.

\CLAC\ regards its input as an unending stream of words separated by spaces. The return key is
interesting to \CLAC\ only because it signals your desire to evaluate the words given so far.
The final result is not affected in any way by the number of lines you use to enter the input.
For example, the single command below would cause \CLAC\ to do exactly the same operations as
the several commands above did.

\begin{verbatim}
     => 2.17 3.14 + 7.23 .41 + + clear
\end{verbatim}

Of course, the ``clear'' at the end will cause \CLAC\ to erase the result before you had a
chance to look at it!

\section{Some Words on the \CLAC\ Command Line}

If you've been following along with \CLAC, you have probably noticed that \CLAC\ does not erase
your last command on the command line. This allows you to easily repeat your last command. For
example you can double the number in stack level one by doing

\begin{verbatim}
     => 2.0 *
\end{verbatim}

This pushes 2.0 onto the stack thereby shifting whatever number was at stack level one to stack
level two. The ``*'' directs \CLAC\ to pop two items, multiply them, and push the result back
onto the stack.

Since \CLAC\ does not erase your last command, you can just tap the return key over and over and
watch the number in stack level one double each time.

The command line has a number of other useful features. In particular, you can edit the last
command. The left and right arrow keys move the cursor along the command. You can use the home
key to jump to the start of the command and the end key to jump to the end of the command.
Typing a Ctrl+Arrow combination allows you to jump from word to word on the command line.
Finally, the Ins key toggles you between text insert mode and text replace mode.

The editing ability is quite handy for correcting mistakes. Suppose you type

\begin{verbatim}
     => 2.17 15.8 + 7.93 +
\end{verbatim}

Let's say that the answer you get looks funny. ``Oops,'' you say as you look at your last
command (which \CLAC\ preserves for you), ``that 2.17 was supposed to be 21.7.'' All you have to
do is type the home key, arrow over to the number, fix it, and type return.

\CLAC\ even remembers the last few command lines. You can use the up and down arrow keys to
scroll through your last eight commands. This lets you review some of your previous
calculations. Also, you can rerun your previous calculations after possibly adjusting them.

By the way, if you type a normal key (ie not an editing key) immediately after running a
command, \CLAC\ will simply erase the command line. We hope you haven't been backspacing over
old command lines to type in new ones!

\section{Back to The Stack}

Here are the commands you must use to get \CLAC\ to do the four basic operations.

\begin{tabbing}
\hspace*{3em}\=+\hspace{3em}\=Adds stack levels 1 and 2.\\
\>             -\>            Subtracts stack level 1 from stack level 2.\\
\>             *\>            Multiplies stack levels 1 and 2.\\
\>             /\>            Divides stack level 2 by stack level 1.\\
\end{tabbing}

All of these operations remove the original values from the stack and replace them with the
single result. Notice that the order \CLAC\ uses for subtraction and division resembles the way
you might write the operations on paper (if, that is, you draw your stacks the way we do). For
example if want to calculate $1234 - 34$ the stack will look like this just before you do the
operation:

\begin{tabular}{|r|} \hline
    0 \\ \hline
    0 \\ \hline
 1234 \\ \hline
   34 \\ \hline
\end{tabular}

\CLAC\ does not allow you to group operations with parenthesis. Although this may sound
surprising, the stack oriented nature of \CLAC\ makes parenthesis totally unnecessary. Consider
the following calculation.

\begin{displaymath}
     \frac { 2.34 + 5.67 }{ 6.78 - 8.98 }
\end{displaymath}

First, calculate the numerator.

\begin{verbatim}
     => 2.34 5.67 +
\end{verbatim}

Next, calculate the denominator.

\begin{verbatim}
     => 6.78 8.98 -
\end{verbatim}

Notice that the numerator is left on stack level 2. How nice. Compute the final result by doing

\begin{verbatim}
     => /
\end{verbatim}

Of course, you could have done the whole thing in one gulp:

\begin{verbatim}
     => 2.34 5.67 + 6.78 8.98 - /
\end{verbatim}

From now on we will calculate all our examples using on \CLAC\ command line. This will save
space in this tutorial. Remember: you can always break a command down into smaller steps. You
can enter each word one at a time.

Here's another example. Suppose you wanted to calculate

\begin{displaymath}
     \frac { 1.64 + \frac {4.52}{3.49 - 6.71} }{ \frac {5.63}{7.82} }
\end{displaymath}

Simple.

\begin{verbatim}
     => 4.52 3.49 6.71 - / 1.64 + 5.63 7.82 / /
\end{verbatim}

In general, you want to work from the innermost subexpression toward the outermost
subexpression. Calculate numerators first. The stack will save as many temporary values as you
need.

Try this:

\begin{displaymath}
     \frac{1.11 + \frac{2.22 + (3.33 + 4.44)/6.66 }{7.77} }{8.88}
\end{displaymath}

Cake.

\begin{verbatim}
     => 3.33 4.44 + 6.66 / 2.22 + 7.77 / 1.11 + 8.88 /
\end{verbatim}

This would be a good time to work through the exercises below. The solutions are at the end of
this tutorial.

\section{Still More Stack Tricks}

Generally you should calculate numerators first. This is because the ``/'' operation expects the
numerator to be in stack level 2. If you calculate the numerator first, that is usually where it
will end up.

Suppose you forget this rule. No problem. \CLAC\ offers a number of words that allow you adjust
the stack. For example ``swap'' exchanges stack levels one and two. This is handy for those
times where you get the numerator and denominator backwards. For example

\begin{displaymath}
     \frac{ 4.45 + 6.77 }{ 9.87 - 2.33 }
\end{displaymath}

This can be calculated using

\begin{verbatim}
     => 9.87 2.33 - 4.45 6.77 + swap /
\end{verbatim}

Of course, with command line editing, you could have just gone back and fixed the mistake. But,
what if you had already committed to the calculation of the denominator?

Suppose you calculate a number into stack level one which is wrong. You want to forget it. The
``drop'' operation pops stack level one and throws it away.

\begin{verbatim}
     => 4.45 6.77 + 9.87 2.23 -
\end{verbatim}

(Oops, that 2.23 was supposed to be a 2.33. Let me recalculate the denominator...)

\begin{verbatim}
     => drop 9.87 2.33 - /
\end{verbatim}

There are several \CLAC\ commands that just rearrange items on the stack. They are often
necessary in complex calculations to move the appropriate numbers into position. It isn't always
easy to plan a large calculation properly from the start.

The simplest such operation is ``drop.'' It merely deletes the item in stack level one and
shifts all the other objects down. It is useful to disposing of a calculated result that you no
longer care about.

The ``dup'' command duplicates the item in stack level one so that both stack levels one and two
are identical. All other items are shifted up one level. This command is useful to avoid typing
in a number more than once should it appear more than once in a calculation.

textsl{Talk about roll, rolld, rot}

\section{Scientific Functions}

\CLAC\ supports all the usual scientific functions. As you might expect, each function takes its
argument from stack level one and returns its result to stack level one. For example

\begin{verbatim}
     => 45.67 sin
\end{verbatim}

computes the sine of 45.67 degrees (assuming \CLAC\ is in degree mode). Here is a list of the
common functions:

\begin{tabbing}

\hspace*{3em}\=sin, cos, tan\hspace{5em}\=Trig functions.\\
\>             asin, acos, atan\>         Inverse trig functions.\\
\>             ln, log\>                  Natural log and base 10 log.\\
\>             exp, exp10\>               $e^{x}$ and $10^{x}$.\\
\>             sqrt, sq\>                 Square root and square.\\

\end{tabbing}

For example, suppose you wanted to calculate the distance between the two points $(4.556,
6.778)$ and $(1.224, -7.936)$ in the cartesian plane. The answer is, of course

\begin{verbatim}
     => 4.556 1.224 - sq 6.778 -7.936 - sq + sqrt
\end{verbatim}

The hyperbolic cosine is given by

\begin{displaymath}
     \cosh (x) = \frac{ e^{x} + e^{-x} }{ 2 }
\end{displaymath}

Let's calculate $\cosh(1.34)$

\begin{verbatim}
     => 1.34 dup exp swap neg exp + 2 /
\end{verbatim}

Supposedly the following is an identity

\begin{displaymath}
     \frac { \sin(A) + \sin(B) }{ \sin(A) - \sin(B) } =
       \frac { \tan((A+B)/2) }{ \tan((A-B)/2) }
\end{displaymath}

Shall we see if this is true for $A = 22.3$ degrees and $B = 37.8$ degrees? First let's get the
left hand side:

\begin{verbatim}
     => 22.3 sin dup 37.8 sin dup neg 4 roll + 3 rolld + swap /
\end{verbatim}

We'll be a little less esoteric about the right hand side by resorting to entering the angles
twice (see if you can do it without entering the angles twice).

\begin{verbatim}
     => 22.3 37.8 + 2 / tan 22.3 37.8 - 2 / tan /
\end{verbatim}

Now the stuff in stack level one and the stuff in stack level two should be the same.

% -------------
% Chapter Break
% -------------
\chapter{Programming}

\CLAC\ is fully programmable. The language we designed for \CLAC\ was inspired by the Forth
computer language and the language used by the Hewlett-Packard HP-48SX calculator. If you are
used to a language such as C or Perl, you will find \CLAC\ programs to be very strange looking.
However, the syntax used by \CLAC\ is a natural consequence of its stack oriented nature. If you
think in terms of the stack, programming \CLAC\ becomes very straightforward.

\CLAC\ can take its input from three sources.

\begin{description}
  
\item[Interactively.] The command line at the bottom of the screen is \CLAC's source of
  interactive words. You can use all of \CLAC's programmable features on the command line. This
  is very useful for writing short loops.
  
\item[From .cpg files.] You can direct \CLAC\ to start reading words from a program file
  (default extension = .cpg) with the ``run'' command. For example

\begin{verbatim}
          => "myprog" run
\end{verbatim}
  
  will execute the program in the file ``myprog.cpg'' Notice that the quotes around ``myprog''
  are not really necessary due to the way \CLAC\ defaults all words to type String that it does
  not recognize.
  
  The text in a \CLAC\ program file can contain any sequence of words understood by \CLAC.
  
\item[From strings.] You can direct \CLAC\ to read words out of a string with the ``eval"
  command. For example

\begin{verbatim}
          => "pi *" eval
\end{verbatim}

     will execute the program contained in the string ``pi *''

\end{description}

These features, coupled with the basic control structures, and \CLAC's associative directory,
make \CLAC\ into a powerful and complete programming language.

\section{Control Structures}

Suppose you wanted to calculate the sum of the first 100 integers. You would use a loop, of
course.

\begin{verbatim}
     => 0 1 100 for I + next
\end{verbatim}

The first zero is the accumulator. The next two numbers represent the loop's range. In this
case, the range is from 1 to 100. The ``for'' word starts the loop. It pops the range values
from the stack thus leaving the zero in stack level 1.

Inside the loop, the word ``I'' means the current loop counter. The first time through the loop
I will be 1. The ``next'' word advances the value of I and loops back to the word just after
``for.'' If it's time for the loop to exit, ``next'' does not loop back, of course, but just
continues on.

The ``next'' does not have to be on the same line as the ``for.'' In fact, you could enter as
many lines as you wanted between a for and its next. You can do arbitrarily complicated
calculations between a for and its next, including other loops.

Suppose you wanted to calculate

\begin{displaymath}
     \frac{1}{2} + \frac{2}{3} + \frac{3}{4} + \frac{4}{5} + \cdots +
       \frac{99}{100}
\end{displaymath}

No problem.

\begin{verbatim}
     => fraction 0 1 99 for I I 1 + / + next
\end{verbatim}

It looks strange. But it works.

Suppose you wanted to calculate the above series often, but with different bounds. One way is to
put the meaty part of the program into a string.

\begin{verbatim}
     => "for I I + / + next" Series sto
\end{verbatim}

Now you can evaluate for various bounds like this

\begin{verbatim}
     => 0 1 99 Series eval
\end{verbatim}

When \CLAC\ sees the word ``Series'' it calls up the string you defined from the directory. The
``eval'' word causes the text of that string to be interpreted as a program. Since the first
word in that program is ``for'', the 1 and 99 on stack levels one and two are taken to be the
limits of the for loop. Notice that it's still necessary to put an initial zero on the stack
outside of the evaluated string. All the eval effectively does is expand the string so that the
sequence of words \CLAC\ sees includes the words in that string.

You can thus use strings for storing quick and dirty programs.

\end{document}
