
\chapter{Design Documentation}

This is the technical documentation for Y. To understand this document, you should be familiar
with C++ programming. This is not a tutorial on that language. Also, you should have access to
the Y source code.

The document also assumes that you are familiar with how to use Y. Please refer to the Y v2.0
Tutorial and Reference for more information. That document describes the execution model of Y
from the user's point of view. You will need to understand that model before you can make much
sense of the implementation described here.

The intent of this document is to put you into a position where you can make meaningful changes
to Y with confidence. In addition, this document points out some aspects of Y's design that you
may find useful in general. Finally, this document explains some of Y's weaknesses.

Originally, Y was designed with the students at Vermont Technical College in mind. For several
semesters in the late 1980s, we at VTC switched editors every semester. We couldn't seem to find
one that addressed the issues we found important. Many public domain editors were not powerful
enough for the more advanced programming classes, and commercial editors were usually overkill,
complicated to install, and could not be copied legally by the students.

My first ideas for a text editor were rather simple. As Y evolved, however, I realized that it
could become powerful enough to use for "serious" editing applications. I therefor decided, when
Y was well under development, that I would try to build an editor capable of satisfying my own
editing needs.

Originally, Y was designed for implementation in C. When I received my copy of Borland
International's Turbo C++ Version 1.0, I decided to change my strategy and implement Y in C++.
Because of the immaturity of my C++ skills when I started Y (and the immaturity of C++ in
general at that time), the C++ code in Y has a very C-like feeling. Over the years the code has
gradually taken on a more idiomatic C++ form but this transformation is still ongoing.

As with any program, the quality of Y's code varies considerably from section to section.
Partially as a result of the above two design changes, some of Y's code is, in fact, not very
well constructed. Indeed, if I had designed Y for implementation in C++ from the beginning,
there are numerous things that I would have done differently. The fact that the development of Y
went fairly smoothly reflects well on the usefulness of C++ in real world applications. Any
programming project will go well if people only attempt well designed programs! C++ worked
nicely even for a not so well designed program.

\section{Layout of the Source Code}

Y supports four different compilation environments.

\begin{enumerate}

\item Visual Studio 2012 on Windows. The 64 bit build is primary but it should be a simple
matter to configure the project files to produce a 32 bit executable as well. Note that at
the time of this writing, Y is not ``64~bit clean.'' Fixing that is a TODO item.

\item Clang v3.3 via Makefiles on Linux. Y is being updated to take advantage of C++ 2011
features. Currently clang v3.3 is the primary C++ 2011 compiler used.

\item Code::Blocks with gcc (or clang) on Linux. The Code::Blocks project files are currently
configured to use gcc, but it should be a simple matter to switch them to clang by providing
a suitable compiler definition.

\item Open Watcom 1.9. Targets for DOS 16 bit, Windows 32 bit, and OS/2 32 bit are supported.
Only the Windows host has been tested but other hosts will probably work fine. Note that
Open Watcom is, at the time of this writing, not able to process any siginificant C++ 2011
features. Thus support for Open Watcom exists only on paper in the hope that Open Watcom will
eventually be upgraded sufficiently in the future.

\end{enumerate}

Using multiple compilers not only allows Y to be built for multiple systems, but it also
improves reliability by exposing the source to various warnings and portability issues.

Y makes use of a screen handling library called Scr and an additional support library called
Spica. These libraries are independent of Y and can be compiled and tested separately. They can
also be used in other projects. The first step in building Y is to build Scr and Spica. The
Visual Studio and Code::Blocks project files do this automatically; for the other supported
compilers it is necessary to build Scr and Spica manually as an initial step.

\section{Tour of the Source}

All of the Y source code is in a single folder. It is roughly divided into four parts.

\begin{enumerate}
\item General purpose utility components.
\item YEditFile classes.
\item Command implementations.
\item Macro handing support.
\end{enumerate}

In the subsections that follow certain key files and systems are described in more detail.

\subsection{EditBuffer}

Class EditBuffer is very important to Y. Objects from this class are strings of arbitrary length
supporting basic editing operations. This class is used not only to implement the lines in the
file being edited, but also the parameters which are manipulated by the parameter handler.

The methods of this class allow, for example, insertion, deletion, and replacement of characters
at arbitrary offsets. If the offset from the beginning of the string is larger than the length
of the string, the string is simply extended with spaces.

This class offers a conversion to char* which allows objects of class EditBuffer to be used in
contexts where char* is expected. Although this is convenient, it is also a potential problem.
When an EditBuffer object is manipulated, the data area where the string is being stored may
be moved. Thus any previously converted char* values will be invalidated.

In addition, this creates problems for future versions of Y that want to swap material to disk.
Ideally, all that would need be done to implement that ability would be to revise the
implementation of class EditBuffer so that swapping is done on a line by line basis. However, in
such a scenario, a char* obtained from an EditBuffer may become invalid if an insertion is done
into ANY EditBuffer object (not just the one which was used to get the char*).

\subsection{List}

Because Y predates the first C++ standard it provides its own list handling support. The List
class implements a generic doubly linked lists of void*. Y specializes this class several
different ways. Class List forms the basis of the list of loaded files, the list of deleted
files, the list of EditBuffers that compose the text of each file, and the list of EditBuffers
that compose the parameter lists.

Each object of class List maintains two sentinel list nodes, a 'Head' and a 'Tail'. These nodes
are never used for data; their 'Data' members are always NULL. However, their presence
simplifies the implementation of many of the methods. The constructor for class List allocates
the sentinel nodes and links them together. The destructor frees all the nodes (but NOT the
things pointed at by the void* data elements).

Each list object maintains a notion of 'current point'. The get() method returns the void*
associated with the current point very quickly. The current point \emph{cannot} be off the head
of list; get() will never return the data member associated with the head sentinel. However, the
current point may be off the tail of the list. In that case, get() will return the NULL pointer.

Each list object also maintains easy access to the total number of elements in the list and the
index of the current point. Asking for this information using the Nmbr\_Items() and
Current\_Index() member functions does not cause the entire list to be scanned.

Class List also offers an operator[](). This function allows the list to be randomly accessed.
This function also positions the current point. Operator[]() is smart in that it tries to use
the shortest path to the desired element. For example, if the index of the current point is 100,
the index of the tail is 200, and the desired new index of the current point is 190,
operator[]() will position the current point to the tail and back it up 10 elements rather than
moving it forward 90 elements.

Class List also offers a Next() and a Previous() member. The Next() member acts something like
*p++. That is, it returns the void* associated with the current point and then advances the
current point. Previous() acts something like *--p. That is, it backs up the current point and
then returns the void* associated with the new current point.

\subsection{FIXME}

\textit{The material below this point is old and probably very out of date.}

WINDOW.CPP

Objects from class Window are simple rectangular colored regions on the screen which remember
their background.

Each file which is loaded into Y is handled internally as an instance from a class derived from
YEdit\_File. Accordingly, class YEdit\_File has a large interface. Operations such as moving the
current point, inserting and deleting text, saving to disk, turning on and off blocks, etc are
all member functions in this large class.

YEdit\_Files, however, know nothing about each other. The fact that Y can load a large number of
files at once is due to the file list. Class File\_List, in the main directory, implements a
single abstract object (ie all of the member functions are static) which binds the many
YEdit\_Files together.

YEdit\_Files also know nothing about the user interface. The command functions defined in the
files COMMAND?.CPP, in the main directory, implement the details of each command. In some cases,
the command functions simply invoke an appropriate member in the YEdit\_File object for the
active file. In other cases, the command functions must do considerable preparatory work. Class
YEdit\_File can, however, print messages by calling the functions in SUPPORT.CPP. All messages
produced by Y go through the functions in this file.

Because YEdit\_File is so large, I built it up using inheritance. In particular, I started with a
base class, called Edit\_File. Objects from class Edit\_File contain the actual text of the file,
a notion of the current point in the file, a record of wether or not the text has been changed,
and some functions for dealing with line oriented blocks.

From Edit\_File I then derived a number of classes which added specialized functions to the
interface of class Edit\_File. For example, class Cursor\_Edit\_File has public functions which
move the current point around, class Disk\_Edit\_File has public functions for reading and writing
the disk, etc. Each of these derived classes have some supporting private data members.

It is important to realize that, except for the fact that they are all derived from Edit\_File,
each of the XXX\_Edit\_File classes are all independent.

Finally, class YEdit\_File is derived, using multiple inheritance, from all of the XXX\_Edit\_File
classes! This merges the public interfaces from all of these classes into one rather large
public interface while at the same time holding the independent concerns of YEdit\_File's various
aspects separate.

Of course, each YEdit\_File object must have only one Edit\_File base. All of the XXX\_Edit\_File
subobjects must be sharing the same text and block information! This is no problem: Edit\_File is
a virtual base class of all of the XXX\_Edit\_File classes.

I think that this is a nice example of multiple inheritance. Even though the interface to
YEdit\_File is large, the interface of each XXX\_Edit\_File is manageable. Furthermore, the
XXX\_Edit\_Files can be used in other contexts to build editors.

For example, suppose you need a simple editor imbedded in another program. Take class Edit\_File
as your base. You probably want to move the current point and insert and delete characters, so
you take Cursor\_Edit\_File and Character\_Edit\_File as well. Do you need to deal with a disk? If
not, then you skip over Disk\_Edit\_File. If you don't need search and replace or word processing
features, skip over Search\_Edit\_File and WP\_Edit\_File. Finally, you take the components you do
need and glue them together with multiple inheritance to get a Edit\_File class which suits your
purposes. I have not tried this, so I don't know how well it would really work.

The virtual base object, from class Edit\_File in this case, is the common ground shared by all
the subobjects in a YEdit\_File object. It is anomalous for class Edit\_File to know about blocks
when there is a Block\_Edit\_File subobject which is supposed to deal with blocks. Normally, this
could be dealt with using virtual functions in the virtual base class. For example, in class
Edit\_File, we might have

\begin{verbatim}
struct Block_Definition {
    long     Start_Line, End_Line;
    unsigned Start_Column, End_Column;
};

virtual Block_Definition Get_Block_Definition();
virtual void Set_Block_Definition(Block_Definition);
virtual boolean          Get_Block_State();
virtual void             Set_Block_State();
\end{verbatim}

These functions would be implemented in Block\_Edit\_File (where they belong) and used by all the
other classes derived from Edit\_File (which need to know about blocks to do their work).

Y currently does not deal with blocks which cover only part of a line. That is, the
implementation of the virtual Get\_Block\_Definition() might force Start\_Column to zero and
End\_Column to UINT\_MAX. You'd like to have it so that fancier blocks could be implemented by
simply adjusting the functions in Block\_Edit\_File.

I started implementing block mode as I've described it above. However, Turbo C++ (version 1.0)
proved to be buggy with regard to virtual functions in virtual base classes. Eventually, I
decided that the code was too unreliable on my compiler and I gave up the attempt. Y currently
has class Edit\_File knowing too much about blocks. It would be a nice exercise to "do blocks
right" for Y.

EDITLIST.CPP

This file implements class Edit\_List. This class is a specialization of class List which deals
with a list of EditBuffer* objects. Notice that this class does not deal directly with the
EditBuffers themselves.

EDITFILE.CPP

This file implements the base class from which class YEdit\_File is built. It contains an object
from class Edit\_List which represents the text of the file. It also contains an object from
class File\_Position which keeps track of the current point in the file.

The current point in the Edit\_List is not related to the current point in the file. In fact,
many of the XXX\_Edit\_File classes have member functions which leave the current point of the
Edit\_List object in an undefined state. Each XXX\_Edit\_File member function which wants the
Edit\_List's current point in a known state must move it there before it tries to use the
Edit\_List.

FILEPOS.CPP

This file implements class File\_Position. Objects from this class deal with a file's current
point. Cursor row and column is maintained as well as the row and column of the display's upper
left corner (and thus the relative position of the cursor on the display).

Objects of this class prevent the cursor from wandering outside the display or the display from
riding up too high or too far to the left. Class Cursor\_Edit\_File allows clients of YEdit\_File
to get directly at the File\_Position member in the Edit\_File base with the CP() member function.
Thus most cursor positioning functions are really just members in class File\_Position. This
behavior also allows for easy saving and restoring of the cursor position. For example

\begin{verbatim}
void f()
  {
    File_Position Old_Position = Active_File.CP();

    // Move Active_File.CP() with calls like:
    //   Active_File.CP().Jump_To_Line(10L);
    // etc...

    Active_File.CP() = Old_Position;
    return;
  }
\end{verbatim}

In the above example, both the cursor position and the relative position of the cursor within
the display are saved and restored.

If you don't like the way Y moves the cursor, you should be able to change only the members of
class File\_Position.

YEDITF.CPP

This is the total YEdit\_File class. It is build from the XXX\_Edit\_File class via multiple
inheritance. In addition, it adds some Y specific features. In particular, member functions for
dealing with the screen, setting colors, etc are defined in class YEdit\_File. In addition
several virtual functions are introduced in class YEdit\_File. These function will be described
later.

Y.CPP

This is the file which contains main(). This file is concerned primarily with opening
activities. The processing of the command line, and the initial reading of FILELIST.YFY is done
here. Main() itself just initializes Y and enters an infinite loop which extracts a keystroke
from the keyboard handler and passes it to a function in this file. That function searches a
table for the proper command function to handle that keystroke. Control thus touches main()
between every keystroke.

COMTABLE.CPP

This file contains the dispatch table for the commands. The keystroke used to activate each
command function is defined here. Thus if you want to reconfigure the keyboard, this is the only
file that you must change.

COMMAND  CPP     7410   4-11-91   8:38p ; Numeric keypad keys
COMMANDA CPP     8401   4-21-91   1:18p ; Alt+Function keys
COMMANDC CPP     8231   4-11-91   9:16p ; Cntrl+Function keys
COMMANDF CPP     5659   4-11-91   9:21p ; Function keys
COMMANDL CPP     2613   4-11-91   9:25p ; Alt+Letter keys
COMMANDS CPP     6606   4-21-91   9:19a ; Shift+Function keys

These file implement the command functions. As you can see, they are organized by invoking
keystroke groups. If you are familiar with Y's keyboard layout (as I assume you are), then you
should have no problem locating a given command function. For example, the function which is
invoked when you try to rename a file is in COMMANDA.CPP since the keystroke used to rename a
file is Alt+F2.

If you want to add a new command, take the following steps:

\begin{enumerate}

\item Decide which keystroke you want to use to invoke your command.

\item Add a command function to the appropriate file. All the command functions have names which
  end with ...Com. For example, "My\_CommandCom()."

\item Edit COMMAND.HPP to include a prototype of your new command function. Note that it must
  look something like boolean My\_CommandCom();

\item Edit COMTABLE.CPP to include an entry for your command function and to associate it with a
  keystroke.

\item ReMAKE the program!

\end{enumerate}

You can call upon a function in class File\_List to get a reference to the currently active
YEdit\_File. For example:

\begin{verbatim}
boolean My_CommandCom()
  {
    YEdit_File &Active_File = File_List::Active_File();

    // Play with the public interface to Active_File.
    // return ERROR if you fail; otherwise return NO_ERROR.
  }
\end{verbatim}

FILELIST. CPP

Class File\_List, implemented in FILELIST.CPP, is an example of a class being used to implement a
single abstract object. In other words, all of class File\_List's members are static: there is
logically only one File\_List. In fact, class File\_List defines a private constructor (which does
nothing) to prevent you from accidently declaring another File\_List object.

Class File\_List deals with the fact that Y can load multiple files. For example, the
\{Next\_File\} and \{Previous\_File\} commands draw on functions in class File\_List. When Y
reloads files after an external program runs, or writes FILELIST.YFY, it is functions in class
File\_List that are at work.

Class File\_List maintains a doubly linked list of YEdit\_File pointers on the heap. Because
File\_List holds pointers to YEdit\_Files rather than YEdit\_Files themselves, the possibility
exists that objects from classes derived from YEdit\_File are really on the list. This is, in
fact, the case. For each file type (ie extension), a different kind of YEdit\_File has been
derived. Class YEdit\_File declares certain functions as virtual and these are overridden for
each of the derived classes.

For example, the Next\_Procedure() function in YEdit\_File is virtual. It's default action, that
is, it's implementation in class YEdit\_File, is to print the message "Can't find procedures in
this file type." In class ASM\_YEdit\_File, however, Next\_Procedure() searches for PROC keywords.

The fact that there are several different types of YEdit\_Files is hidden inside of the
implementation of File\_List. Everywhere else in the program there are only pointers (or
references) to YEdit\_Files. The virtual functions insure consistent behavior.

If you want to add another file type, simply derive a new class from YEdit\_File and override
whichever of the virtual functions you need to override. You will also need to edit
File\_List::New\_File(). It is in that function where the particular type of YEdit\_File is
created. In the body of File\_List::New\_File() there is a switch statement, the only one of its
kind in the program, which has a case for each type of YEdit\_File. You will need to add a new
case.

If you want another behavior to be file type specific, simply make the appropriate member
function of YEdit\_File virtual. You can then override that function in whichever derived classes
you want. For the other YEdit\_File types, you do nothing; the default virtual function in
YEdit\_File (the original behavior) will step in automatically. Likewise, the command functions
do not need to be edited.

SPECIAL.CPP

This file contains the implementations of the specialized YEdit\_Files. That is, the virtual
functions in class YEdit\_File which are being overridden by the derived classes are implemented
in this file. In particular, it is in this file that you will find the Next\_Procedure() and
Previous\_Procedure() functions for the various file types.

HELP.CPP

This file contains the help screens which Y displays. I wrote the help text using WordPerfect,
version 4.2. You can find the "source" files for the help text in the DOC directory. The program
TXTTOC, in the TOOLS directory, converts a text file (such as produced by directing Wordperfect
to print to a file) into a C language declaration of an array of character pointers.

KEYBOARD.CPP

This file implements the abstract keyboard handler object. This object finds the next keystroke
and returns it to main() for dispatching.

Keystrokes can come from many sources. For example, they can come from the keyboard itself, from
a keyboard macro, or from a set of repeated keystrokes. Y does not really care where keystrokes
come from. Thus the keyboard handler can hide all manner of interesting and unusual things. In
Version 2.0 of Y I may implement a full macro language. My plan is to do so by elaborating on
the keyboard handler.

Inside the keyboard handler, an abstract base class called Keyboard\_Script is defined. From this
abstract class, several concrete class are derived with one such class for each "source" of
keystrokes. For example class NeverEnding\_Source reads keystrokes directly from the keyboard.
Class Keyboard\_Macro, however, takes its keystrokes out of an array.

The keyboard handler sets up a stack of Keyboard\_Script* with the address of a
NeverEnding\_Source object at the bottom. Whenever a caller wants a keystroke, the keyboard
handler consults the currently active Keyboard\_Script on the stack. Most of the time this
results in reading a key from the keyboard. If, however, the key is a special control character,
the keyboard handler builds an appropriate object and puts it on the stack. For example, the \^R
character causes the keyboard handler to build a Repeat\_Sequence object.

Each Keyboard\_Script class will return -1 when they have no more keystrokes to give. In such a
case, the keyboard handler will pop the stack and try to draw another keystroke out of the new
Keyboard\_Script object now on the top of the stack. NeverEnding\_Source objects, naturally, never
return -1.

Although this may sound complicated, it has two nice features.
\begin{enumerate}
\item Giving Y the ability to read keystrokes from other places is easily integrated into the
  existing program.

\item All kinds of interesting combinations work perfectly. For example, repeatedly executing a
  keyboard macro which contains a repeat sequence would involve a stack like this

\begin{verbatim}
TOP  Current Script -> Repeat_Sequence
                       Keyboard_Macro
                       Repeat_Sequence
BOTTOM                 NeverEnding_Source
\end{verbatim}
\end{enumerate}

If you wish to give Y the ability to deal with multiple keyboard macros which can invoke each
other, it should be fairly straightforward.

I would like to point out that this is, I think, a nice example of using C++'s polymorphic
abilities to create a heterogenous stack.

Quoted characters are handled in a rather gross way. If the keyboard handler see a \^Q
character, it draws another keystroke out of the currently active Keyboard\_Script and turns on
the most significant bit of that key code. The command dispatching function back in Y.CPP checks
for this special case. If it finds it, the keystroke is unconditionally passed to
Insert\_LetterCom().

This is gross because it means that the keyboard handler knows that certain key codes are
available for use. On the other hand, it integrates nicely with the structure I described above.
For example, the sequence \^R10\^Q\^E causes a Repeat\_Sequence object to be built which will
return the \^E code with its most significant bit ON ten times.

PARA.CPP

This file implements class Parameter. Objects from this class manage an input box and an
Edit\_List of eight parameters (EditBuffers). There is an instance from this class for each of
the different input boxes Y provides.

Class Parameter contains a static member Insert(). This function takes a string as a parameter
and copies it into a special buffer in PARA.CPP. When any parameter handler object is asked to
come up with a parameter, the "inserted" string is used (once). This allows parameters to be
"force fed" to a command. For example

\begin{verbatim}
Parameter::Insert("CONFIG.SYS");
Find_FileCom();
\end{verbatim}

causes the file CONFIG.SYS to be loaded into the editor. Normally Find\_FileCom() will prompt the
user for a filename. However, when it asks its parameter handler object for a parameter the
string "CONFIG.SYS" will be returned instead.

This ability will be very important when a macro language is implemented.
