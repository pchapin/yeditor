\chapter{Introduction}

\textit{This document is very incomplete. Furthermore some of the information presented here
  about Y version 2.0 may not yet be implemented. While I hope you find this document useful, at
  the current time you may need to refer to Y's source code for definitive information about Y's
  behavior.}

Y is a macro driven programmer's text editor. This document describes Y 2.0. Version 2.0 is a
major upgrade from the previous version 1.1 that is documented elsewhere. A short section of
this document discusses some of the important ways that version 2.0 is different than version
1.1.

Y was designed with the following criteria in mind:

\begin{enumerate}

\item It had to be useful in the programming classes at Vermont Technical College (VTC). It had
  to be simple enough for the beginning student to learn quickly yet powerful enough for the
  more advanced student to use on complex programming tasks.

\item It had to be cross platform. In some classes at VTC we use embedded versions of DOS where
  a powerful yet relatively compact text editor is necessary. In addition many VTC classes use
  Windows or Linux as their platform of choice and students, like anyone else, want to be able
  to use the same editor everywhere.

\item It had to run satisfactorily on an 80386 class machine and be usable on ``dumb''
  terminals.

\item It had to be open source and extremely easy to install. VTC students would be encouraged
  to make copies and they had to be able to do so with minimal legal or technical complications.

\item It had to be small enough to load quickly from an embedded hard drive or from a busy
  network drive.

\item It had to allow the editing of about 200,000 bytes (or more) of source code at once, even
  in the memory limited DOS environment.

\end{enumerate}

Here is what Y does:

\begin{enumerate}

\item Y operates on text files in a natural way for manipulating the source code of programs.
  Many commonly used operations (such as deleting a line or switching to a new file) can be done
  with one keystroke.

\item Y can load and manipulate many files at the same time.

\item Y can use external programs or Perl scripts to manipulate the files (or portions of the
  files) you are editing. You can effectively extend Y by writing external programs to do
  complex tasks that are not built into the editor.

  This is an extremely powerful feature. Y does not, for example, support a sorting option
  because you can use an external sorting utility (such as the one that comes with your
  operating system) to sort text in Y. Similarly, Y does not need to support regular expression
  search and replaces because you can use a Perl script to perform such operations.

\item Y supports a full featured macro language that allows you to implement new, non-trivial,
  interactive features. For example, you can implement a line-draw mode, new command line
  options, special formating features, and many other things using the macro language. Y does
  not need all these features built in because you can add exactly what you need yourself and
  put your new features into a macro library file.

  In addition, Y will read a startup macro when it is first invoked. The startup macro can be
  used to change the editor's personality by associating arbitrary macro commands with any key
  on the keyboard. You can also perform arbitrary text editing and even terminate Y in the
  startup macro. Thus you can think of the startup macro as a program interpreted by Y. This
  allows you to use Y non-interactively as a scripting language for doing complex batch editing
  tasks.

\item Y supports features that make it useful in a variety of unusual environments. You can use
  ANSI escape sequences to do the screen redrawing and plain ASCII keystrokes for all the
  commands. This allows Y to be used in contexts where all special keys and graphic features may
  not be available such as through a BBS door, over a plain terminal, or via a telnet client.

\end{enumerate}

Here is what Y does \emph{not} do:

\begin{enumerate}

\item Y does not do word processing. Y is a programmer's editor. While it can do rudimentary
  word processing (especially with the right macros) Y is not, and never will be suited for the
  construction of large text documents without the aid of a separate document processor such as
  troff or \TeX.

\item Y does not do disk swapping. This is not an issue on virtual memory operating systems such
  as Windows, OS/2, or Unix. However in the DOS environment Y's lack of disk swapping support
  puts an upper limit on the amount of text that can be edited. Y is designed to allow for a
  useful amount of text in this context but how true that is will depend on your application.

\item Y does not do backups. Editors that making backup files or that ask for confirmation
  before they save anything are annoying. Those extra backup files merely fill the disk with
  something to delete. Those "pretty" dialog boxes merely serve to interrupt one's
  concentration. I agree that good backup practices are important, but making backups is not
  something you can do with a text editor. Use a backup program.

\end{enumerate}

This document contains three parts: a tutorial on using Y effectively, a reference manual for Y,
and a description of Y's internal architecture and design. These parts are intended to address,
respectively, the needs of new and casual Y users, power users, and developers wishing to
contribute to Y.

I designed the tutorial for both beginning and advanced users. It starts by describing how to
load files and do simple editing tasks. In later sections, I describe how you can write special
purpose programs to extend Y using either a traditional programming language, Perl, or custom
macros.

I recommend that you try out the steps being discussed as you read the tutorial. You should
prepare an empty directory or disk to use as you work through the tutorial.

Comments, bug reports, etc are welcome. Please send them to

\begin{verbatim}
     Peter C. Chapin
     Vermont Technical College
     Williston, VT 05495
     PChapin@vtc.vsc.edu
\end{verbatim}
