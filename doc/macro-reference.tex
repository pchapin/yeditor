\chapter{Macro Reference}

Whenever a key is pressed an associated macro is pushed onto the macro stack and executed. Even
simple keys such as letter keys invoke a macro (usually \texttt{add\_text}) to handle that
keystroke. Since Y 2.0 is still under development the precise macros associated with particular
keys is subject to change. Thus for information on the current associations, please refer to the
Y source code\ldots\ specifically the file \texttt{WordSource.cpp}. Toward the bottom of that
file is an array \texttt{keyboard\_map} that maps key codes to the macro text that is executed
when that key is pressed.

Note that the macro associated with a key can be changed interactively or as part of macro
execution using the ``define\_key'' macro command. Thus the associations in
\texttt{WordSource.cpp} should be taken as defaults or initial values only.

When using the ``define\_key'' macro command you need to provide the name of the key you are
defining. The official names (case sensitive) are given in the array \texttt{key\_names} near
the bottom of \texttt{WordSource.cpp} in the Y source code. Notice that decimal names for the
``extended ASCII'' keys are used rather than hex names because you can use decimal values on the
numeric keypad to enter those keystrokes.

The functions in the editor source code that implement each macro command (and ultimately
document each macro command) are specified in the file \texttt{command\_table.cpp}. At the top
of that file the array \texttt{command\_table} maps command name to the C++ function that
implements that command. Again, since Y 2.0 is under development the precise mappings from macro
command to C++ function are subject to change.

Note that there may be macro commands that are not bound to any key; that is, some macro
commands may not appear in the \texttt{keyboard\_map} mentioned previously. These commands can
be bound to keys or executed directly in user defined macros.

The C++ command functions themselves are stored in files such as \texttt{command\_X.cpp} where
\texttt{X} is the first letter of the command function's name. For example the ``add\_text''
macro command is implemented by the \texttt{add\_text\_command} function found in
\texttt{command\_a.cpp}.

More information about the macro language will be forthcoming in this document as time allows.
