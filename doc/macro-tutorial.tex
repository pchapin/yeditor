\chapter{Macro Tutorial}

A Y macro is a string of macro text. Internally, Y maintains a stack of these macro strings. The
string in stack level one is the currently executing macro. If that macro invokes another macro,
the new macro is pushed into stack level one and the suspended macro is moved to stack level
two. When the new macro completes execution, it is popped and the suspended macro returns to
stack level one so that it can continue execution.

Each key on the keyboard is associated with some macro text. When Y starts the macros associated
with each key are set to Y's initial defaults. However, using the define\_key macro command, you
can redefine the macro text associated with a key. Typically the start up macro executes many
define\_key commands to remap the keyboard as the user sees fit.

Since you can bind an arbitrarily complex macro to a key with define\_key, the keyboard
remapping is not just a simple matter of moving built in commands to different keys. You can
actually construct new commands using the built in ones or previously defined macros and bind
your new commands to appropriate keys.

Since define\_key is just another macro command, you can define macros that redefine keys on the
fly. For example, to implement a line draw mode, you could write a macro that uses define\_key
to bind new functions to the arrow keys.

Ordinarily, Y takes its commands from the macro text currently on the macro stack. Only if the
macro stack is empty does Y consult the keyboard. When you press a key on the keyboard, Y pushes
the macro text associated with that key onto the macro stack and executes it in the usual way.
If that macro text invokes other macros they are pushed onto the macro stack and executed as
well. Only when the macro associated with the keystroke is completely finished and the macro
stack is empty again does Y return to the keyboard for its next command.

One consequence of this is that even the letter keys have macros associated with them. By
default these macros merely add the corresponding letter to the file currently being edited.
However, if you want certain letters to exhibit special behavior, it's a straight forward matter
to redefine the associated macro. The special handling of `\{' and `\}' in .C files is
implemented in this way.

The macro language itself is also stack based. In addition to the macro stack that holds the
text of the pending macros, Y also maintains a parameter stack that holds the strings upon which
the macro commands operate. Macro text is read from left to right and is regarded as a sequence
of white space delimited words. Each word enclosed in quotation marks is pushed onto the
parameter stack as it is encountered. Words that are built in macro commands or other named
macros are executed as they are encountered. Words that are not recognized are assumed to be
strings and pushed onto the parameter stack. Thus Y does not require a string to be enclosed in
quotation marks if the string is not the name of a built in command or existing macro and if the
string contains no embedded spaces. This is convenient for dealing with numbers and certain
other things, but I don't recommend making it a practice of leaving out the quotation marks.

For example, the following macro inserts the text ``Y v2.0!'' into the current file:

\begin{verbatim}
"Y v2.0!" add_text
\end{verbatim}

Y pushes the first item onto the stack since it is a quotation enclosed string. Notice that it
is pushed as a single word despite the embedded space. Y recognizes the second item as a macro
command that pops a string off the stack and inserts it into the current file.

As another example, here is a macro that inserts ``()'' into the file and then moves the cursor
to the left so that it will be just after the `(' character.

\begin{verbatim}
"()" add_text cursor_left
\end{verbatim}

By default the macros associated with the letter keys are very simple. Here is the default macro
associated with the `Y' key (distinct from the `y' key).

\begin{verbatim}
"Y" add_text
\end{verbatim}

Because macros can define other macros the need often arises to nest quoted strings. Within a
quoted string, you can use the `$\backslash$' character to temporarily disable the meaning of
the quotation mark. For example, the following macro defines the `Y' key so that when pressed
the text ``Y v2.0!'' is inserted into the file instead of just `Y'

\begin{verbatim}
K_Y "\"Y v2.0!\" add_text" define_key
\end{verbatim}

The define\_key command takes two string arguments. In stack level two (pushed first), it
expects the name of the key to define. Since ``K\_Y'' is not the name of a built in command or
an existing macro, Y will push ``K\_Y'' onto the stack. It happens that K\_Y is meaningful to
the define\_key command. The define\_key command then looks at the material in stack level one
to find the macro to associate with the specified key. Since the macro itself contains quotation
marks, the macro that executes the define\_key has to deal with nested quotes.

Nesting quotes works to arbitrary depth. Here is a macro that will bind to the ALT+0 key a macro
that, when invoked, will redefine the `Y' key. In other words this macro sets things up so that
ALT+0 remaps the keyboard.

\begin{verbatim}
K_ALT0 "K_Y \"\\\"Y v2.0!\\\" add_text\" define_key" define_key
\end{verbatim}

This works because the `$\backslash$' character also temporarily disables the special meaning of
`$\backslash$'.

As you can see setting up such nested macros correctly is difficult. To address this problem, Y
allows you to quote strings a different way. The `\{$\ldots$\}' characters can be used like
quotation marks. For example

\begin{verbatim}
{Y v2.0!} add_text
\end{verbatim}

is exactly the same as

\begin{verbatim}
"Y v2.0!" add_text
\end{verbatim}

There are a few differences, however. Extra white space is squeezed out of \{$\ldots$\}
delimited strings. That is the following macro inserts all the spaces exactly as indicated.

\begin{verbatim}
" a     b " add_text
\end{verbatim}

but the macro below only inserts ``a b'' into the file without the spaces

\begin{verbatim}
{ a     b } add_text
\end{verbatim}

This two line macro also only inserts ``a b'' into the file

\begin{verbatim}
{ a
  b } add_text
\end{verbatim}

In addition, comments are possible inside \{$\ldots$\} delimited strings. Y macro comments start
with a `\#' character and run to the end of the line. Compare:

\begin{verbatim}
"# This is not a comment" add_text
\end{verbatim}

with

\begin{verbatim}
{ a   # This is a comment!
  b } add_text
\end{verbatim}

The other significant difference between \{$\ldots$\} and ``$\ldots$'' is that the curly braces
nest more easily. Here is that macro that defines ALT+0 so that when pressed, the `Y' key is
remapped.

\begin{verbatim}
K_ALT0 { K_Y { "Y v2.0!" add_text } define_key } define_key
\end{verbatim}

The start up macro that Y executes normally contains a large number of define\_key commands.
Since the macro text associated with certain keys might be very complex, the ability to use
\{$\ldots$\} enclosed strings to break up the macro text over several lines and to intermix
comments is absolutely essential.

All built in macro commands that take parameters take string parameters from the parameter
stack. In some cases, the strings are converted into numbers. What happens if the parameter
stack is empty? Just as when the macro stack is empty, Y ``goes interactive'' in such a case.
Each built in command will pop up an input box and let the user enter whatever text they want.
That text is then immediately pushed onto the parameter stack and the macro command is allowed
to pop that text in the usual way.

For example, the find\_file command expects to find the name of a file to load on the parameter
stack. The macro associated with the F1 key (Y v1.1 compatibility mode) is just ``find\_file.''
Since no parameter is given, find\_file will pop up an input box when the user selects F1.

However, a macro such as

\begin{verbatim}
"C:\\AUTOEXEC.BAT" find_file
\end{verbatim}

will cause Y to load \filename{C:$\backslash$AUTOEXEC.BAT}. Notice that you need to disable the
special meaning of `$\backslash$' in the string given to find\_file so that the input word
recognizer puts the appropriate string onto the parameter stack.

The find\_file command has a default prompt that it uses when it goes interactive. If you want
to use a different prompt, you can use the ``input'' macro command. Input takes its prompt from
the parameter stack and pushes the string entered by the user back onto the parameter stack. For
example

\begin{verbatim}
"Backup File:" input find_file
\end{verbatim}

With respect to macro processing, Y supports the execute\_macro and execute\_file commands. The
execute\_macro command pops the parameter stack and executes the string it finds there as a
macro (pushing it onto the macro stack). The execute\_file command pops the parameter stack for
a file name and then pushes the contents of the file into stack level one of the macro stack for
execution. By binding these commands to appropriate keys, you can enter macro text or run files
of macro text interactively. For example:

\begin{verbatim}
K_ALTM { execute_macro } define_key
\end{verbatim}

The error\_message command allows you to display error messages to the user. The text popped
from stack level one of the parameter stack is put into an error box underneath the word
"Sorry." For example

\begin{verbatim}
K_ALTM { "Unable to execute!" error_message } define_key
\end{verbatim}

You can experiment with what error messages look like by binding the error\_message command
without a parameter to the key of your choice. For example, if you have execute\_macro bound to
the ALT+M key, select it and in the input box popped up by execute\_macro, type

\begin{verbatim}
K_ALTE { error_message } define_key
\end{verbatim}

Then type ALT+E to get error\_message to go interactive so you can enter test messages from the
keyboard.
